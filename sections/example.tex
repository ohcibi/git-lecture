\section{Beispiel mit zwei Entwicklern}
\subsection{Bowling Game Kata}
\begin{frame}
  \frametitle{Bowling Game Kata}
  \tableofcontents[currentsection,currentsubsection]
\end{frame}
\begin{frame}
  \frametitle{Bowling Game Kata}
  \begin{itemize}
    \item Coding Kata zur t�glichen �bung
      \begin{itemize}
        \item http://www.butunclebob.com/ArticleS.UncleBob.TheBowlingGameKata
      \end{itemize}
    \item Ziel: Testgetriebene Entwicklung einer Punktez�hl-Logik f�rs Bowling
    \item Lernziel: Perfekte Aus�bung der einzelnen notwendigen Schritte zur L�sung des Problems (\textbf{Nicht} die L�sung des Problems!)
    \item �bungsform: regelm��ige Wiederholung
      \begin{itemize}
        \item Variante: Ausf�hrung im Takt zu Musik
      \end{itemize}
    \item "`Kata"' Analogie zur Kampfkunst
  \end{itemize}
\end{frame}
\begin{frame}
  \frametitle{Bowling-Exkurs}
  \begin{itemize}
    \item Ein Spiel besteht aus 10 Frames.
    \item Jeder Frame besteht aus 2 W�rfen.
    \item Pro Frame werden die gefallenen Pins als Punkte gutgeschrieben.
    \item Bei einem Spare (10 Pins mit zwei W�rfen in einem Frame) wird der \emph{n�chste Wurf} doppelt gez�hlt.
    \item Bei einem Strike (10 Pins mit einem Wurf - das beendet den
      Frame) werden die \emph{n�chsten zwei W�rfe} doppelt gez�hlt.
    \item Die Regeln f�r den letzten Frame sind irrelevant f�r die Kata (Erweiterungen sind willkomen).
  \end{itemize}
\end{frame}
\begin{frame}
  \frametitle{Tests f�r die Kata}
  \begin{itemize}
    \item "`Gutter Game"' (20 W�rfe mit jeweils 0 Pins)
    \item "`All Ones"' (20 W�rfe mit jeweils einem Pin)
    \item "`One Spare"' (Ein Spare, ein Wurf mit weniger als 10 Pins und sonst 0-Runden)
    \item "`One Strike"' (Ein Strike zwei W�rfe mit zusammen weniger als 10 Pins und sonst 0-Runden)
  \end{itemize}
\end{frame}
